\documentclass[]{article} % use larger type; default would be 10pt
\usepackage[utf8]{inputenc} % set input encoding (not needed with XeLaTeX)

%no indented paragrapsh!
\setlength{\parindent}{0in}
\setlength{\parskip}{0.2in}

%%% Examples of Article customizations
% These packages are optional, depending whether you want the features they provide.
% See the LaTeX Companion or other references for full information.
%%% PAGE DIMENSIONS
\usepackage{geometry} % to change the page dimensions
\geometry{a4paper} % or letterpaper (US) or a5paper or....
\geometry{top=0.5in}
\usepackage{graphicx,color} % support the \includegraphics command and options
%primary color palette

%%% PACKAGES
\usepackage{amsmath}
\usepackage{booktabs} % for much better looking tables
%\usepackage[]{fullpage} 
\usepackage{array} % for better arrays (eg matrices) in maths
\usepackage{paralist} % very flexible & customisable lists (eg. enumerate/itemize, etc.)
\usepackage{verbatim} % adds environment for commenting out blocks of text & for better verbatim
\usepackage{subfig} % make it possible to include more than one captioned figure/table in a single float
\usepackage{multirow} 
\usepackage{rotating}
\usepackage{amsthm}


\begin{document}

\section{Converting $x$ and $y$ Coordinates into Fresnel Integrals}
First, we need to convert the Euler version into the Fresnel version.  We start with the $x$ coordinate first.

The Euler version with constant of proportionality equal to $\alpha$ is given by
\[
	x(s) = \int_{0}^{s} \cos \frac{\alpha u^2}{2} du
\]
and we have that the standard Fresnel version is 
\[
	C(z) = \int_{0}^{z} \cos \frac{\pi t^2}{2} dt
\]
Apply the following transformation $t = \frac{\sqrt{\alpha}}{\sqrt{\pi}} u$ we have $du = \frac{\sqrt{\pi}}{\sqrt{\alpha}} dt$ and then
\begin{eqnarray*}
	x(s) &=& \int_{0}^{s} \cos \frac{\alpha u^2}{2} du \\
	     &=& \frac{\sqrt{\pi}}{\sqrt{\alpha}} \int_{0}^{\frac{\sqrt{\alpha}}{\sqrt{\pi}} s} \cos \frac{\pi t^2}{2} dt \\
	     &=& \frac{\sqrt{\pi}}{\sqrt{\alpha}} C\left(\frac{\sqrt{\alpha}}{\sqrt{\pi}} s\right)
\end{eqnarray*}
The same holds the second integral for the $y$ coordinate.
\[
	y(s) = \frac{\sqrt{\pi}}{\sqrt{\alpha}} S\left(\frac{\sqrt{\alpha}}{\sqrt{\pi}} s\right)
\]
where 
\[
	S(z) = \int_{0}^{z} \sin \frac{\pi t^2}{2} dt
\]

\section{Calculating Angle}
Second we need to calculate the angle at the tangency point. In particular we cant to know when we have turned through $\frac{\pi}{2}$ the first time.

Be we know from the definition of the Euler spiral that 
\begin{eqnarray*}
	\frac{dy}{dx} &=& \tan \left( \frac{\alpha s^2}{2}\right) \\
				  &=& \tan \left( \frac{\pi}{2}\right)
\end{eqnarray*}
which implies $s^*$ the arc length when the angle turned is $\frac{\pi}{2}$ must satisfy 
\[
	s^* = \frac{\sqrt{\pi}}{\sqrt{\alpha}}
\]
Substituting into the equations above we get
\begin{eqnarray*}
x(s^*) &=& \frac{\sqrt{\pi}}{\sqrt{\alpha}} C\left( 1 \right) \\
y(s^*) &=& \frac{\sqrt{\pi}}{\sqrt{\alpha}} S\left( 1 \right) \\
\end{eqnarray*}

\section{Matching The Radius}
We want to make sure that the radius matches the height when at the apex i.e. when we have turned through $\frac{\pi}{2}$.
\begin{eqnarray*}
	r &=& y(s^*) \\ 
	  &=& \frac{\sqrt{\pi}}{\sqrt{\alpha}} S\left( 1 \right)
\end{eqnarray*}
which gives
\[
	\alpha = \frac{\pi^2}{r^2} S(1) ^2
\]
\end{document}
